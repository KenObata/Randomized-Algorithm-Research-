\def\year{2021}\relax
%File: formatting-instructions-latex-2021.tex
%release 2021.1
\documentclass[letterpaper]{article} % DO NOT CHANGE THIS
\usepackage{aaai21}  % DO NOT CHANGE THIS
\usepackage{times}  % DO NOT CHANGE THIS
\usepackage{helvet} % DO NOT CHANGE THIS
\usepackage{courier}  % DO NOT CHANGE THIS
\usepackage[hyphens]{url}  % DO NOT CHANGE THIS
\usepackage{graphicx} % DO NOT CHANGE THIS
\usepackage{amsmath}% Added
\usepackage{nameref}% Added
\urlstyle{rm} % DO NOT CHANGE THIS
\def\UrlFont{\rm}  % DO NOT CHANGE THIS
\usepackage{natbib}  % DO NOT CHANGE THIS AND DO NOT ADD ANY OPTIONS TO IT
\usepackage{caption} % DO NOT CHANGE THIS AND DO NOT ADD ANY OPTIONS TO IT
\frenchspacing  % DO NOT CHANGE THIS
\setlength{\pdfpagewidth}{8.5in}  % DO NOT CHANGE THIS
\setlength{\pdfpageheight}{11in}  % DO NOT CHANGE THIS
%\nocopyright
%PDF Info Is REQUIRED.
% For /Author, add all authors within the parentheses, separated by commas. No accents or commands.
% For /Title, add Title in Mixed Case. No accents or commands. Retain the parentheses.
\pdfinfo{
/Title (AAAI Press Formatting Instructions for Authors Using LaTeX -- A Guide)
/Author (AAAI Press Staff, Pater Patel Schneider, Sunil Issar, J. Scott Penberthy, George Ferguson, Hans Guesgen, Francisco Cruz, Marc Pujol-Gonzalez)
/TemplateVersion (2021.1)
} %Leave this
% /Title ()
% Put your actual complete title (no codes, scripts, shortcuts, or LaTeX commands) within the parentheses in mixed case
% Leave the space between \Title and the beginning parenthesis alone
% /Author ()
% Put your actual complete list of authors (no codes, scripts, shortcuts, or LaTeX commands) within the parentheses in mixed case.
% Each author should be only by a comma. If the name contains accents, remove them. If there are any LaTeX commands,
% remove them.

% DISALLOWED PACKAGES
% \usepackage{authblk} -- This package is specifically forbidden
% \usepackage{balance} -- This package is specifically forbidden
% \usepackage{color (if used in text)
% \usepackage{CJK} -- This package is specifically forbidden
% \usepackage{float} -- This package is specifically forbidden
% \usepackage{flushend} -- This package is specifically forbidden
% \usepackage{fontenc} -- This package is specifically forbidden
% \usepackage{fullpage} -- This package is specifically forbidden
% \usepackage{geometry} -- This package is specifically forbidden
% \usepackage{grffile} -- This package is specifically forbidden
% \usepackage{hyperref} -- This package is specifically forbidden
% \usepackage{navigator} -- This package is specifically forbidden
% (or any other package that embeds links such as navigator or hyperref)
% \indentfirst} -- This package is specifically forbidden
% \layout} -- This package is specifically forbidden
% \multicol} -- This package is specifically forbidden
% \nameref} -- This package is specifically forbidden
% \usepackage{savetrees} -- This package is specifically forbidden
% \usepackage{setspace} -- This package is specifically forbidden
% \usepackage{stfloats} -- This package is specifically forbidden
% \usepackage{tabu} -- This package is specifically forbidden
% \usepackage{titlesec} -- This package is specifically forbidden
% \usepackage{tocbibind} -- This package is specifically forbidden
% \usepackage{ulem} -- This package is specifically forbidden
% \usepackage{wrapfig} -- This package is specifically forbidden
% DISALLOWED COMMANDS
% \nocopyright -- Your paper will not be published if you use this command
% \addtolength -- This command may not be used
% \balance -- This command may not be used
% \baselinestretch -- Your paper will not be published if you use this command
% \clearpage -- No page breaks of any kind may be used for the final version of your paper
% \columnsep -- This command may not be used
% \newpage -- No page breaks of any kind may be used for the final version of your paper
% \pagebreak -- No page breaks of any kind may be used for the final version of your paperr
% \pagestyle -- This command may not be used
% \tiny -- This is not an acceptable font size.
% \vspace{- -- No negative value may be used in proximity of a caption, figure, table, section, subsection, subsubsection, or reference
% \vskip{- -- No negative value may be used to alter spacing above or below a caption, figure, table, section, subsection, subsubsection, or reference

\setcounter{secnumdepth}{0} %May be changed to 1 or 2 if section numbers are desired.

% The file aaai21.sty is the style file for AAAI Press
% proceedings, working notes, and technical reports.
%

% Title

% Your title must be in mixed case, not sentence case.
% That means all verbs (including short verbs like be, is, using,and go),
% nouns, adverbs, adjectives should be capitalized, including both words in hyphenated terms, while
% articles, conjunctions, and prepositions are lower case unless they
% directly follow a colon or long dash

\title{Randomized Candidate Voting Methods for Preventing Manipulation}
\author{
    %Authors
    % All authors must be in the same font size and format.
    Kenichi Obata\thanks{Supervised by Valerie King, Nishant Mehta}
    %Written by AAAI Press Staff\textsuperscript{\rm 1}\thanks{With help from the AAAI Publications Committee.}\\
    %AAAI Style Contributions by Pater Patel Schneider,
    %Sunil Issar, 
    \\
}
\affiliations{
    %Afiliations

    %\textsuperscript{\rm 1}Association for the Advancement of Artificial Intelligence\\
    %If you have multiple authors and multiple affiliations
    % use superscripts in text and roman font to identify them.
    %For example,

    % Sunil Issar, \textsuperscript{\rm 2}
    % J. Scott Penberthy, \textsuperscript{\rm 3}
    % George Ferguson,\textsuperscript{\rm 4}
    % Hans Guesgen, \textsuperscript{\rm 5}.
    % Note that the comma should be placed BEFORE the superscript for optimum readability
    University of Victoria\\
    % email address must be in roman text type, not monospace or sans serif
    kobata@uvic.ca
    % See more examples next
}
\iffalse
%Example, Single Author, ->> remove \iffalse,\fi and place them surrounding AAAI title to use it
\title{My Publication Title --- Single Author}
\author {
    % Author
    Author Name \\
}

\affiliations{
    Affiliation \\
    Affiliation Line 2 \\
    name@example.com
}
\fi

\iffalse
%Example, Multiple Authors, ->> remove \iffalse,\fi and place them surrounding AAAI title to use it
\title{My Publication Title --- Multiple Authors}
\author {
    % Authors

        First Author Name,\textsuperscript{\rm 1}
        Second Author Name, \textsuperscript{\rm 2}
        Third Author Name \textsuperscript{\rm 1} \\
}
\affiliations {
    % Affiliations
    \textsuperscript{\rm 1} Affiliation 1 \\
    \textsuperscript{\rm 2} Affiliation 2 \\
    firstAuthor@affiliation1.com, secondAuthor@affilation2.com, thirdAuthor@affiliation1.com
}
\fi
\begin{document}

\maketitle

\begin{abstract}
This paper investigates the problem of voting manipulation under complete information when manipulators know a full preference profile of other voters. Our main contribution includes analysis of a randomized candidate method as a potential solution to prevent voting manipulation and probabilistic measurement of the approximation of the randomized algorithm.
\end{abstract}

\section{Introduction}
The Gibbard-Satterthwaite theorem proves that any deterministic voting rule cannot simultaneously satisfy the following three properties.\\
i) Any alternative can be elected. \\
ii) Not dictatorial: no single player can determine the winner of the game. \\
iii) Strategy-proof: no player is better-off by misrepresenting their preference \cite{Satterthwaite}.\\
Hence, we consider a randomized voting method as a potential solution to the voting manipulation.\\
A brief description of the randomized candidate method introduced by Bentert and Skowron is the following. Given a set of candidates $C$ with size $m$, with $n$ number of voters, we first fix a candidate subset size, $l \le m$, to be assigned to each voter. Each voter gives the linear order of preference. We assume no voter knows this candidate assignment of other voters, as well as score results of each candidate. In the end, we compute the total score the candidate received, divided by the number of times each candidate is ranked. The final score is n times the weighted score, and a candidate with the highest score will be elected. The score computed from the algorithm is under condition where all $m$ candidates are ranked at least one time. However, since $Pr(c \in C \text{ never ranked}| \text{ n voters}) = (\frac{m-l}{m})^n$,  this is exponentially small when n is large.\\
Given the randomized candidate method, our main contribution is the analysis of the algorithm to investigate followings: \\
i) The randomized candidate algorithm reduces the chances that a manipulator has an incentive to misrepresent their preference. ii) Consequences of manipulation. iii) Noise of the randomized candidate method. 

\section{Related Work}
While Bentert and Skowron's research motivation is to approximate deterministic voting methods with less information to be efficient, this paper analyzes the randomized algorithm as a potential solution to the voting manipulation. Their results show that even $l = 2$, with hundreds of voters, we can approximate minimal variance \cite{Skowron}.\\
Procaccia analyzes the approximation of the lottery extension, selecting a winner based on the Borda count distribution. While the lottery extension is strategy-proof, it has an upper bound of approximation $\frac{1}{2} +\Omega{ \frac{1}{\sqrt{m}} }$  \cite{Procaccia_1}.\\
Veselova introduces the concept of a manipulator having an $incentive$ to misrepresent their preference under deterministic voting methods. Veselova defines that if there exists at least one possible situation when manipulation makes him better off, and nothing changes in all other situations, then the manipulator has an incentive to play strategically \cite{Veselova}. We extend the concept of having an incentive to randomized voting methods defined in the Preliminaries section.\\
Ayadi et al. demonstrate a comparison of scoring voting methods Their research asks each voter to rank only top-k candidates. Their empirical result shows that harmonic scoring approximates better than Borda count \cite{Ayadi}. Since Bentert and Skowron prove that the randomized algorithm approximates well having large n with any decreasing order of the scoring method, we do not specify a scoring method in this paper. %However, it will be beneficial to investigate the best scoring method in the randomized candidate method as a future study.

%\section{Our contribution}
%Our main contribution is the analysis of the randomized candidate method. We investigate followings: \\
%i) The randomized candidate algorithm reduces the chances that a manipulator has an incentive to misrepresent their preference. \\
%ii) Consequences of manipulation.\\
%iii) Noise of the randomized candidate method. 

%\section{The Randomized Candidate Voting Method} 
%\label{Algorithm}
%A brief description of the randomized candidate method introduced by Bentert and Skowron is the following. Given a set of candidates $C$ with size $m$, with $n$ number of voters, we first fix a candidate subset size, $l \le m$, to be assigned to each voter. Each voter gives the linear order of preference. We assume no voter knows this candidate assignment of other voters, as well as score results of each candidate. In the end, we compute the total score the candidate received, divided by the number of times each candidate is ranked. The final score is n times the weighted score, and a candidate with the highest score will be elected. The score computed from the algorithm is under condition where all $m$ candidates are ranked at least one time. However, since $Pr(c \in C \text{ never ranked}| \text{ n voters}) = (\frac{m-l}{m})^n$,  this is exponentially small when n is large.

\section{Preliminaries}
%Since we investigate constructive manipulation, let A be the manipulator’s target candidate. In other words, he schemes to make A elected.
We first suppose a manipulator knows the preference profile of other voters. Secondly, the manipulator schemes to do constructive manipulation, which they want to make their most favorite candidate, $A$, elected.\\
Since each voter $v_i$ is equipped with a linear order of preference over the candidates, $pos_{v_i(}c)$ means the position of a candidate $c$ in $v_i$’s preference ranking.\\
Based on the randomized candidate method, Bentert and Skowron, denote random variable $X_c$ as the score that candidate c receives from the algorithm. Since we assume a manipulator has complete information of other voters, they can compute E[$X_c$] of all candidates, which implies that they can construct an aggregated ranking of all candidates based on E[$X_c$]. To analyze the algorithm let us denote $C_{higher}$ as a set of candidates having higher E[$X_c$] than that of the manipulator’s target candidate $A$. Similarly, denote $C_{lower}$  as a set of candidates having lower E[$X_c$] than that of the manipulator’s target candidate $A$.%or $A$ itself
\\Bentert and Skowron show that the expected score from the randomized candidate algorithm can be expressed as follows: \\ 
\begin{equation} 
E[X_c] = \frac{1}{\binom{m-1}{l-1}}\sum_{\text{all voter}} \sum_{i=1}^l \alpha_i \binom{pos_v(c)-1}{i-1}\binom{m-pos_v(c)}{l-i} 
\end{equation}
where $\alpha$ is a score vector, a decreasing order of points to be given to each candidate. A common scoring method is Borda rule, defined by $B(c) = m - pos(c)$ $\forall c \in C$. \\
Adding to those terminologies, we extend Veselova's definition of $incentive$ to our case that a manipulator has an incentive to misrepresent their preference under randomized candidate method when they can achieve the following inequality: \\
There exists at least one $higher \in C_{higher}$ such that
\begin{equation}
E[X_{higher} \mid \text{misrepresent preference}] < E[X_{higher} \mid truthful]\nonumber 
\end{equation}
$\text{ and for all other } higher \in C_{higher},$
\begin{equation}
E[X_{higher} \mid \text{misrepresent preference}] \le E[X_{higher} \mid truthful]
\end{equation}
%$\text{ and for all other } higher \in C_{higher}.$
 %$$\forall higher \in C_{higher}.$$
 % We might need to move this down to analysis section because we haven't talked about pos(A) up means higher E[X_c].
For example, given a set of candidates = (A, B, C, D, E), suppose a manipulator’s true preference is ($A\succ B\succ C\succ D\succ E$). We suppose aggregated ranking of E[$X_c$] is ($B\succ C\succ A\succ D\succ E$) where B has the highest E[$X_c$]. For $l=2$, if candidate assignment is (C, D), then the manipulator has an incentive to misrepresent their preference. Since truthful voting is $C\succ D$, and misrepresented vote is $D \succ C$, the inequality $(2)$ is satisfied. 
%Hence, the manipulator has an incentive to manipulate in this situation.

\section{Chances of losing an incentive to manipulate}
A manipulator needs aggregated ranking information to manipulate and increase the chances that their favorite candidate $A$ wins. Under deterministic voting methods, aggregated ranking is produced based on the total score of each candidate. Hence we consider how a manipulator would obtain an aggregated ranking under the randomized method. Ideallly, they need the conditional probability of winning for each candidate under truthful voting. Although computing the probability is still work in progress, a manipulator can still obtain aggregated ranking based on $E[X_c]$ of each candidate $c \in C$, and they scheme to satisfy the inequality $(2)$.
%First, when n is small, a manipulator can compute each candidate's conditional probability of being elected under their truthful voting. Then aggregated ranking can be produced based on the likelihood. It can be computed by counting the number of ways $X_A > X_c$
%$\forall c \in C \symbol{92}\text{\{} A\text{\}}$ divided by the total ways of candidate assignment for all voters, $\binom{m}{l}^n$. To count the number of ways $X_A$ is the highest, it is necessary to verify $\binom{m}{l}^n $ cases. This implies that since $\binom{m}{l}$ is always an integer, the runtime of computing probabilities of each candidate being elected is exponential time. Thus, a manipulator being able to compute the exact probabilities of each candidate with a conventional computer is only possible when n is small.\\
%However, this does not mean a manipulator stops misrepresenting their preference when n is large. As Bentert and Skowron prove, the probability that the score computed by the randomized candidate method differs from its expected score is upper bounded by $2\cdot\exp{-\delta^2E[X_c]/3}$ for $\delta \in [0,1]$. The randomized candidate method's actual score would be close to the $E[X_c]$ if n is large. Hence, the manipulator can obtain aggregated ranking based on $E[X_c]$ of each candidate $c \in C$, and they scheme to satisfy the inequality $(2)$.
%Hence, the manipulator schemes to maximize $E[X_c]$ for his target candidate A and decrease $E[X_{higher}\mid\text{manipulation}]$ $\forall higher \in C_{higher}$ from  $E[X_{higher}\mid \text{truthful}]$. 
Hence, we have the following result.
\begin{equation}
\begin{split}
Pr(\text{a manipulator loses an incentive to manipulate})\\
= \frac{\binom{pos(A)}{l}+\binom{m-pos(A)+1}{l}}{\binom{m}{l}}
\end{split}
\end{equation}
$\binom{pos(A)}{l}$ refers to the number of ways that candidates $\in C_{higher}$ and $A$ occupy all the assignment. If $A$ is in the subset, the manipulator always ranks $A$ top. If $l = 2$, their votes is always $A\succ c \in C_{higher}$, which is thier truthful voting. If $l>2$, no matter how the manipulator orders the assigned candidates, they cannot satisfy the inequality $(2)$ without violating the inequality for other candidates in $C_{higher}$. This is same when A is not in the subset.
$\binom{m-pos(A)+1}{l}$ refers to the number of ways that candidates rank lower than A and A occupy all the assignment. In this situation, the manipulator has no control to change the $E[X_c]$ $\forall c \in X_{higher}$. Intuitively speaking, in this case, the manipulator cannot reduce the gap of score between their favorite candidate and candidates in the higher position. These cases are mutually exclusive. Hence, we sum these cases and divide by the total ways choosing $l$ subset of candidates from $m$ candidates. This explains the randomized candidate method can reduce the manipulator’s incentive to manipulate with certain probability. 
%This probability can also be applied to the case when n is small. If n is small, a manipulator will compute probabilities of wins for each candidate. In other words, these probabilities are aggregated ranking of the candidates.

\section{Manipulators have a price to pay}
%Suppose a candidate $c_1 \in C_{higher}$ and a candidate$c_2 \in C_{lower} $are assigned to a manipulator, and he ranks ...
Another lemma of the randomized candidate algorithm is that a manipulation increases the expected score of a candidate c $\in C_{lower}$ and not their target candidate A. This is due to the dependency of $E[X_c]$ on $pos(c)$, clearly from the equation (1). Let $pos(c \mid \text{manipulation})$ be a position of c in $C_{lower} \symbol{92}\text{\{} A\text{\}}$ by manipulation and $pos(c\mid \text{truthful})$ be the position by truthful preference. We know $pos(c \mid \text{manipulation}) < pos(c\mid \text{truthful})$. \\
The equation (1) shows that $ \frac{\binom{pos_v(c)-1}{i-1}\binom{m-pos_v(c)}{l-i}}{\binom{m-1}{l-1}}$ is the probability that the candidate c is ranked at position $i$. Since manipulation causes $pos(c \mid \text{manipulation}) < pos(c\mid \text{truthful})$, the probability that the candidate c is ranked at lower i will be increased for $pos(c \mid \text{manipulation})$, which will increase $E[X_c]$ for $c \in C_{lower}$. Moreover, since $\sum_{i=1}^l Pr(\text{c is ranked at i}) =1$,  the probability that the candidate c is ranked at higher i will be decreased accordingly.\\ As manipulation does not change $pos(A)$, the manipulator cannot increase $E[X_A]$.


%I might omit the following details:
%From (1), consider $\binom{pos(c')-1}{i-1}$ and $\binom{pos(c)-1}{i-1}$. These binomial expressions become 0 $\forall i > pos(c’)$, pos(c) respectively. So for each $i= 1,2,...,pos(c')$,  the expression is a non-zero integer. Since $pos(c’) < pos(c)$,  the number of terms in $\sum_{i=1}^{pos(c')} \binom{pos(c')-1}{i-1}$ is less than that of $\sum_{i=1}^{pos(c)} \binom{pos(c)-1}{i-1}$ ...(a). Similarly, consider $\binom{m-pos(c)}{l-i}$. The binomial expression becomes 0 $\forall i$ such that $l-i < m-pos(c)$ where $l,m$ are fixed. Since $pos(c’) < pos(c)$, $m-pos(c’) > m-pos(c) \Rightarrow $ $\binom{m-pos(c)}{l-i} > \binom{m-pos(c')}{l-i}$ for the last index of $i$...(b).\\
%From (a) and (b), the range of the score vector $\alpha =  (a_1, a_2, ..., 1_{l-1})$ multiplied to the non-zero $\frac{\binom{pos(c')-1}{i-1} \binom{m-pos(c')}{l-i} }{\binom{m-1}{l-1}}$ is higher than that of $\frac{\binom{pos(c)-1}{i-1} \binom{m-pos(c)}{l-i} }{\binom{m-1}{l-1}}$. Since $\sum_{i=1}^{l} \frac{\binom{pos(c')-1}{i-1} \binom{m-pos(c')}{l-i}}{\binom{m-1}{l-1}} = 1$, the range of the score vector multiplied to $\sum_{i=1}^{l} \frac{\binom{pos(c')-1}{i-1} \binom{m-pos(c')}{l-i}}{\binom{m-1}{l-1}}$ is higher than $\sum_{i=1}^{l} \frac{\binom{pos(c)-1}{i-1} \binom{m-pos(c)}{l-i}}{\binom{m-1}{l-1}}$.


\section{Noise of the randomized method}
While the randomized candidate method can reduce chances that manipulator has an incentive to manipulate, it has a cost of adding noise. Dubhashi and Ranjan prove that when random variables are negatively associated, we can still apply chernoff bounds \cite{BRICS}. A voter $v_i$ ranks a candidate in $pos_i(c)$ is the model of balls and bins, and hence $X_c$ is negatively associated. Bentert and Skowron demonstrate that $\forall$ $\delta$ $\in$ [0,1], $Pr(|X_c-E[X_c]| \geq \delta\cdot E[X_c]) \leq 2\cdot\exp{-\delta^2E[X_c]/3} $ \cite{Skowron}. This imply that if n is small, noise can be certain probability. \\
For example, for $\delta = 0.1$,  suppose candidate A ranks the top for all voters, then $pos(A) = 1$ for all voters. Using Borda method, $E[X_A]=n\cdot((l-1)\cdot \frac{\binom{m-1}{l-1}}{\binom{m-1}{l-1}}) = n\cdot(l-1).$ Then $Pr(|X_c-E[X_c]| \geq 0.1\cdot E[X_c]) \leq 2\cdot\exp{-0.01n(l-1)/3} $.
With $n=200$ voter and $l=4$, $Pr(|X_c-E[X_c]| \geq 0.1\cdot E[X_c]) \leq 2\cdot\exp{-6/3} = 27\%$.


\section{Conclusion and Future Direction}
As we show already, the randomized candidate method can reduce the chances of manipulator having an incentive to misrepresent their preference. This result means the randomized candidate method could produce strategy-proof voting with some probability. However, if the number of voters is small, the algorithm adds the cost of noise.\\
A possible direction of this study includes three things. First, this paper analyses the case when constructive manipulation with one target. Hence, it would be beneficial to explore more cases when a manipulator has multiple target of constructive manipulation. Analysis1 would be updated.
%find ways to reduce noise that a randomized algorithm creates, actual score deviating from the expected score with some probability. Cloning voters and increase the size of n might be a potential solution.
Secondly, there are multiple positional scoring methods, such as Borda and Harmonic weighting. Although chances that a manipulator has an incentive to misrepresent their preference does not depend on scoring methods, one scoring method could be better than the other in terms of the size of noise. Lastly, exploring the destructive manipulation is also a possible future study.

%\section{Conclusion}

\section{Acknowledgement}
I would like to thank Professor Valerie King and Professor Nishant Mehta for their continued mentorship and advising throughout this project.



\bibliography{bibfile1}


\end{document}
